\documentclass[12pt]{article}

\usepackage[utf8]{inputenc}
\usepackage[spanish,activeacute]{babel}
\usepackage{graphicx}
\usepackage[labelfont=bf]{caption}

\begin{document}
\begin{center}
 {\Large \textbf{Tarea 4}}\\
 Daniel Rozo -201426137\\
\end{center}
\section{Punto 1}
\noindent Los resultados obtenidos para el calculo de la trayectoría del misil tanto con una inclinacion inicial de 45 grados como a distintos angulos se muestran a continuacion:

\begin{centering}
\captionsetup{type=figure}
\includegraphics[height=0.45\textwidth]{45grados.jpg}
\captionof{figure}{Trayectoria del proyectil para un angulo inicial de 45 grados\\}
\label{fig:grado}
\end{centering}


\begin{centering}
\captionsetup{type=figure}
\includegraphics[height=0.45\textwidth]{grados.jpg}
\captionof{figure}{Trayectoria del proyectil para diferentes angulos iniciales\\}
\label{fig:grados}
\end{centering}

\noindent Como se logra observar en la Figura \ref{fig:grado} el misil alcanza una altura de mas de 2000 m y una distancia de mas de 9000 m. Estos dos valores corresponde a la mayor altura que puede alcanzar el misil con la menor distancia. Esto se puede ver al observar la Figura \ref{fig:grados}. Pues para angulos menores a 45 grados se alcanza mayor altura pero la distancia recorrida en x es menor mientras que para los angulos mayores la distancia en x es mayor pero la altura es menor. Cabe observar que el efecto de la friccion sobre la trayectoria del misil no afecta la geometria de esta pues sigue siendo un tiro parabolico. 

\section{Punto 2}

Con respecto al problema de difusion de calor se realizo este con tres condiciones de frontera. La primera de estas es con fronteras fijas a una temperatura 10\degree C. Dado que la temperatura es fija en los bordes y en centro de la varilla se espera que en el limeite cuando el sistema se encuentre en equilibrio igua haya un gradiente de temperatua a lo largo de 




\end{document}
